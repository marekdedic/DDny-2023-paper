\begin{abstract}{english}
	Machine learning algorithms on graphs, in particular graph neural networks, became a popular framework for solving various tasks on graphs, attracting significant interest in the research community in recent years. As presented, however, these algorithms usually assume that the input graph is fixed and well-defined and do not consider the problem of constructing the graph for a given practical task. This work proposes a methodical way of linking graph properties with the performance of a GNN solving a given task on such graph via a surrogate regression model that is trained to predict the performance of the GNN from the properties of the graph dataset. Furthermore, the GNN model hyper-parameters are optionally added as additional features of the surrogate model and it is shown that this technique can be used to solve the practical problem of hyper-parameter tuning. We experimentally evaluate the importance of graph properties as features of the surrogate model with regards to the node classification task for several common graph datasets and discuss how these results can be used for graph composition tailored to the given task. Finally, our experiments indicate a significant gain in the proposed hyper-parameter tuning method compared to the reference grid-search method.

\end{abstract}

\begin{keywords}{english}
	Graph neural network, Graph properties, Meta-learning, Hyper-parameter optimization
\end{keywords}

\begin{abstract}{czech}
	Algoritmy strojového učení na grafech, zejména grafové neuronové sítě, se staly populárním nástrojem pro řešení nejrůznějších úloh na grafech a v posledních letech přitahují značný zájem vědecké komunity. V prezentované podobě však tyto algoritmy obvykle předpokládají, že vstupní graf je pevně daný a dobře definovaný, a neuvažují problém konstrukce grafu pro danou aplikační úlohu. Tato práce navrhuje metodický způsob propojení vlastností grafu s efektivitou GNN řešící danou úlohu na daném grafu prostřednictvím náhradního regresního modelu, který je naučen předpovídat efektivitu GNN modelu z vlastností grafového datasetu. Hyperparametry modelu GNN mohou navíc být přidány jako další příznaky náhradního modelu a je ukázáno, že tuto techniku lze použít k řešení problému optimalizace hyperparametrů. Experimentálně vyhodnocujeme význam jednotlivých vlastností grafu jako příznaků náhradního modelu s ohledem na úlohu klasifikace uzlů pro několik běžných grafových datových sad a diskutujeme, jak lze tyto výsledky využít pro konstrukci grafu přizpůsobenou dané úloze. Naše experimenty ukazují na významný přínos navrhované metody ladění hyperparametrů ve srovnání s referenční grid-search metodou.
\end{abstract}

\begin{keywords}{czech}
	Grafová neuronová síť, Vlatnosti grafů, Meta-learning, Optimalizace hyperparametrů
\end{keywords}
